\begin{comment}
\end{comment}

\makeatletter   % Permet d'accèder aux variables @

%=============================================================================%

    \iftoggle{useCustomFonts}
    {   
        % Deux prochaines lignes -> Pagella et Mathpazo
        \usepackage{mathpazo} % utilise Palatino pour les mathématiques (mettre en premier)
        \usepackage{tgpagella} % utilise la police TeX Gyre Pagella

        % Deux prochaines lignes -> New Century et Fourier
        %\usepackage{newcent}
        %\usepackage{fouriernc}
    }
    {}

%=============================================================================%
    
    \iftoggle{AuteureFemme}
    { \newcommand{\monsieurMadame}{Mme.} }    
    { \newcommand{\monsieurMadame}{M.} }

%=============================================================================%

    \iftoggle{MemoirePasThese}
    {   % Si c'est un mémoire
        \newcommand{\documentPresente}{Mémoire présenté}
        \newcommand{\leDocument}{le mémoire}
        \newcommand{\leGrade}{maître ès science (M.Sc.)}
    }
    {   % Si c'est une thèse
        \newcommand{\documentPresente}{Thèse présentée}
        \newcommand{\leDocument}{la thèse}
        \newcommand{\leGrade}{docteur ès science (Ph.D.)}
    }

%=============================================================================%
    
    % Gestion de la version électronmique vs celle imprimée.
    \newtoggle{oneside}     % Truc pour avoir un else à \if@twoside
    \toggletrue{oneside}
    \if@twoside
        \togglefalse{oneside}
    \fi

    \iftoggle{oneside}
    {   % S'il y a un côté, on fait la version électronique.
        \geometry{letterpaper, lmargin=1.25in, rmargin=1.25in,
                  tmargin=1.5in, bmargin=1.0in}
        % Prochaines trois lignes enlève l'entête
        \renewcommand{\chaptermark}[1]
            {\markboth{{\thechapter. #1}}{}}
        \renewcommand{\sectionmark}[1]{}
    }
    {   % S'il y a deux côtés, on fait la version imprimée!

        % On enlève la numérotation des pages vides
        \let\origdoublepage\cleardoublepage
        \newcommand{\clearemptydoublepage}{%
            \clearpage
            {\pagestyle{empty}\origdoublepage}%
            }
        \let\cleardoublepage\clearemptydoublepage
 
        % Chapitre à la page de droite avec une page gauche blanche
        \let\stdchapter\chapter
        \renewcommand*\chapter{%
              \@ifstar{\starchapter}{\@dblarg\nostarchapter}}
              \newcommand*\starchapter[1]{\clearpage\null\thispagestyle{empty}\stdchapter*{#1}}
              \def\nostarchapter[#1]#2{\clearpage\null\thispagestyle{empty}\stdchapter[{#1}]{#2}}

        % On utilise un fontsize plus petit pour un livre (10pt vs 12pt).
        \let\small\relax
        \let\footnotesize\relax
        \let\scriptsize\relax
        \let\tiny\relax
        \let\large\relax
        \let\Large\relax
        \let\LARGE\relax
        \let\huge\relax
        \let\Huge\relax
        \input{size10.clo}
    }

%=============================================================================%

\makeatother    % Plus d'accès aux variables @
