\begin{comment}
\end{comment}

\makeatletter   % Permet d'accèder aux variables @

%=============================================================================%

    \iftoggle{useCustomFonts}
    {   
        % Deux prochaines lignes -> Pagella et Mathpazo
        \usepackage{mathpazo} % utilise Palatino pour les mathématiques (mettre en premier)
        \usepackage{tgpagella} % utilise la police TeX Gyre Pagella

        % Deux prochaines lignes -> New Century et Fourier
        %\usepackage{newcent}
        %\usepackage{fouriernc}
    }
    {}

%=============================================================================%
    
    \iftoggle{AuteureFemme}
    { \newcommand{\monsieurMadame}{Mme.} }    
    { \newcommand{\monsieurMadame}{M.} }

%=============================================================================%

    \iftoggle{MemoirePasThese}
    {   % Si c'est un mémoire
        \newcommand{\documentPresente}{Mémoire présenté}
        \newcommand{\leDocument}{le mémoire}
        \newcommand{\leGrade}{maître ès science (M.Sc.)}
    }
    {   % Si c'est une thèse
        \newcommand{\documentPresente}{Thèse présentée}
        \newcommand{\leDocument}{la thèse}
        \newcommand{\leGrade}{docteur ès science (Ph.D.)}
    }

%=============================================================================%
    
    % Gestion de la version électronmique vs celle imprimée.
    \iftoggle{VersionLivre}
    {   % On fait la version imprimée!

        % On utilise un fontsize plus petit(10pt vs 12pt).
        \let\small\relax
        \let\footnotesize\relax
        \let\scriptsize\relax
        \let\tiny\relax
        \let\large\relax
        \let\Large\relax
        \let\LARGE\relax
        \let\huge\relax
        \let\Huge\relax
        \input{size10.clo}  % Ajuste le fontsize ET les marges
        \geometry{twoside=true} % L'ordre est important, pour les marges/fontsize.
        %\widowpenalty5000  % Décommenter s'il y a trop de ligne veuves/orphelines.
        %\clubpenalty5000   


        % On enlève la numérotation des pages vides
        \let\origdoublepage\cleardoublepage
        \newcommand{\clearemptydoublepage}{%
            \clearpage
            {\pagestyle{empty}\origdoublepage}%
            }
        \let\cleardoublepage\clearemptydoublepage
        
        % Permet de mettre un page blanche seulement dans la version imprimée
        \newcommand{\autoPageBlancheLivre}{\clearpage\null\thispagestyle{empty}}

        \iftoggle{LivrePGChVide}
        {
            % Chapitre à la page de droite avec une page gauche blanche
            \let\stdchapter\chapter
            \renewcommand*\chapter{%
                  \@ifstar{\starchapter}{\@dblarg\nostarchapter}}
                  \newcommand*\starchapter[1]
                                          {
                                            \clearpage\null\thispagestyle{empty}
                                            \stdchapter*{#1}
                                          }
                  \def\nostarchapter[#1]#2
                                        {
                                            \clearpage\null\thispagestyle{empty}
                                            \stdchapter[{#1}]{#2}
                                        }
        \renewcommand{\autoPageBlancheLivre}{}
        }{}

        % Les hyperliens n'ont pas desoins d'être colorés
        \hypersetup{hidelinks}
        % On affiche les DOI dans la biblio -> Pratique en version imprimée
        \bibliographystyle{nature-fr-showdoi}
    }
    {   % S'il y a un côté, on fait la version électronique.
        \geometry{letterpaper, lmargin=1.25in, rmargin=1.25in,
                  tmargin=1.5in, bmargin=1.0in, twoside=false}
        % Prochaines trois lignes enlève l'entête
        \renewcommand{\chaptermark}[1]
            {\markboth{{\thechapter. #1}}{}}
        \renewcommand{\sectionmark}[1]{}

        % On n'affiche pas les DOI dans la biblio -> Plus propre
        \bibliographystyle{nature-fr}
        %\bibliographystyle{unsrt-fr} % unsrt partiellement traduit. Préférer nature.
        
        % Permet de mettre un page blanche seulement dans la version imprimée
        \newcommand{\autoPageBlancheLivre}{}
    }


%=============================================================================%

\makeatother    % Plus d'accès aux variables @
